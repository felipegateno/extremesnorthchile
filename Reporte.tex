% Options for packages loaded elsewhere
\PassOptionsToPackage{unicode}{hyperref}
\PassOptionsToPackage{hyphens}{url}
%
\documentclass[
]{article}
\usepackage{amsmath,amssymb}
\usepackage{iftex}
\ifPDFTeX
  \usepackage[T1]{fontenc}
  \usepackage[utf8]{inputenc}
  \usepackage{textcomp} % provide euro and other symbols
\else % if luatex or xetex
  \usepackage{unicode-math} % this also loads fontspec
  \defaultfontfeatures{Scale=MatchLowercase}
  \defaultfontfeatures[\rmfamily]{Ligatures=TeX,Scale=1}
\fi
\usepackage{lmodern}
\ifPDFTeX\else
  % xetex/luatex font selection
\fi
% Use upquote if available, for straight quotes in verbatim environments
\IfFileExists{upquote.sty}{\usepackage{upquote}}{}
\IfFileExists{microtype.sty}{% use microtype if available
  \usepackage[]{microtype}
  \UseMicrotypeSet[protrusion]{basicmath} % disable protrusion for tt fonts
}{}
\makeatletter
\@ifundefined{KOMAClassName}{% if non-KOMA class
  \IfFileExists{parskip.sty}{%
    \usepackage{parskip}
  }{% else
    \setlength{\parindent}{0pt}
    \setlength{\parskip}{6pt plus 2pt minus 1pt}}
}{% if KOMA class
  \KOMAoptions{parskip=half}}
\makeatother
\usepackage{xcolor}
\usepackage[margin=1in]{geometry}
\usepackage{graphicx}
\makeatletter
\newsavebox\pandoc@box
\newcommand*\pandocbounded[1]{% scales image to fit in text height/width
  \sbox\pandoc@box{#1}%
  \Gscale@div\@tempa{\textheight}{\dimexpr\ht\pandoc@box+\dp\pandoc@box\relax}%
  \Gscale@div\@tempb{\linewidth}{\wd\pandoc@box}%
  \ifdim\@tempb\p@<\@tempa\p@\let\@tempa\@tempb\fi% select the smaller of both
  \ifdim\@tempa\p@<\p@\scalebox{\@tempa}{\usebox\pandoc@box}%
  \else\usebox{\pandoc@box}%
  \fi%
}
% Set default figure placement to htbp
\def\fps@figure{htbp}
\makeatother
\setlength{\emergencystretch}{3em} % prevent overfull lines
\providecommand{\tightlist}{%
  \setlength{\itemsep}{0pt}\setlength{\parskip}{0pt}}
\setcounter{secnumdepth}{-\maxdimen} % remove section numbering
\ifLuaTeX
\usepackage[bidi=basic]{babel}
\else
\usepackage[bidi=default]{babel}
\fi
% get rid of language-specific shorthands (see #6817):
\let\LanguageShortHands\languageshorthands
\def\languageshorthands#1{}
\usepackage{bookmark}
\IfFileExists{xurl.sty}{\usepackage{xurl}}{} % add URL line breaks if available
\urlstyle{same}
\hypersetup{
  pdflang={esp},
  hidelinks,
  pdfcreator={LaTeX via pandoc}}

\author{}
\date{\vspace{-2.5em}}

\begin{document}

{
\setcounter{tocdepth}{2}
\tableofcontents
}
\section{1. Descripción de la
estación}\label{descripciuxf3n-de-la-estaciuxf3n}

La estación pluviométrica COPIAPO, administrada por DGA, se localiza en
la Región DE ATACAMA, a una latitud de -27.36° y longitud -70.34°, con
una elevación aproximada de 385 m s.n.m.. La estación registra la
variable precipitación diaria (mm/día) y cuenta con una serie de datos
que se extiende desde el 1971-01-01 hasta el 2025-05-31 , totalizando
19875 días en el período de análisis. De estos, 18643 días presentan
registros válidos, lo que corresponde a una cobertura del 93.8\%, sin
detección de valores atípicos.

\section{2. Descripción metodológica de las métricas de
precipitación}\label{descripciuxf3n-metodoluxf3gica-de-las-muxe9tricas-de-precipitaciuxf3n}

La caracterización de la variabilidad, intensidad y extremos de la
precipitación se realizó a partir de la serie diaria de precipitación,
calculando métricas anuales por año hidrológico (abril--marzo). Las
métricas empleadas corresponden a índices ampliamente utilizados en
climatología e hidrología, en particular aquellos propuestos por el
Expert Team on Climate Change Detection and Indices (ETCCDI).

\subsubsection{Precipitación total anual en días húmedos
(PRCPTOT)}\label{precipitaciuxf3n-total-anual-en-duxedas-huxfamedos-prcptot}

La métrica \textbf{PRCPTOT} corresponde a la suma anual de la
precipitación diaria registrada en días húmedos, definidos como aquellos
con precipitación estrictamente mayor a 0 mm. Este indicador permite
cuantificar la cantidad total de agua precipitada en un año hidrológico,
excluyendo los días secos.

Matemáticamente, se define como:

\[
PRCPTOT_j = \sum_{i=1}^{N_j} RR_{ij}, \quad \text{con } RR_{ij} > 0
\]

donde \(RR_{ij}\) es la precipitación diaria del día \(i\) en el año
hidrológico \(j\), y \(N_j\) corresponde al número total de días del
año.

\begin{center}\rule{0.5\linewidth}{0.5pt}\end{center}

\subsubsection{Índice de intensidad diaria simple
(SDII)}\label{uxedndice-de-intensidad-diaria-simple-sdii}

El \textbf{SDII (Simple Daily Intensity Index)} representa la intensidad
media de la precipitación durante los días húmedos de un año
hidrológico. Se calcula como el cociente entre la precipitación total
anual en días húmedos y el número de días húmedos del año.

Se modifico para los años con registro de precipitación 0, para estos se
les asigna un \(SDII = 0\) \[
SDII_j = \frac{\sum RR_{ij}}{N_{wet,j}}, \quad \text{con } RR_{ij} > 0
\]

donde \(N_{wet,j}\) es el número de días húmedos del año hidrológico
\(j\). Valores elevados de SDII indican una mayor concentración de la
precipitación en eventos intensos.

\begin{center}\rule{0.5\linewidth}{0.5pt}\end{center}

\subsubsection{Precipitación máxima en 1 día
(Rx1day)}\label{precipitaciuxf3n-muxe1xima-en-1-duxeda-rx1day}

La métrica \textbf{Rx1day} corresponde al mayor monto de precipitación
diaria registrado dentro de un año hidrológico y es un indicador directo
de eventos extremos de corta duración.

\[
Rx1day_j = \max(RR_{ij})
\]

\begin{center}\rule{0.5\linewidth}{0.5pt}\end{center}

\subsubsection{Precipitación máxima en 5 días consecutivos
(Rx5day)}\label{precipitaciuxf3n-muxe1xima-en-5-duxedas-consecutivos-rx5day}

La métrica \textbf{Rx5day} representa el máximo acumulado de
precipitación en cualquier período móvil de cinco días consecutivos
dentro de un año hidrológico. Para su cálculo se utiliza una suma móvil
de cinco días.

\[
Rx5day_j = \max \left( \sum_{k=i}^{i+4} RR_{kj} \right)
\]

Este índice es relevante para evaluar eventos persistentes asociados a
crecidas y saturación del suelo.

\begin{center}\rule{0.5\linewidth}{0.5pt}\end{center}

\subsubsection{Número de días con precipitación mayor o igual a 10 mm
(R10mm)}\label{nuxfamero-de-duxedas-con-precipitaciuxf3n-mayor-o-igual-a-10-mm-r10mm}

El índice \textbf{R10mm} corresponde al número de días en un año
hidrológico en los que la precipitación diaria es igual o superior a 10
mm, y permite evaluar la frecuencia de eventos de precipitación moderada
a intensa.

\[
R10mm_j = \sum_{i=1}^{N_j} I(RR_{ij} \ge 10)
\]

donde \(I(\cdot)\) es una función indicadora que toma valor 1 cuando la
condición se cumple y 0 en caso contrario.

\begin{center}\rule{0.5\linewidth}{0.5pt}\end{center}

\subsubsection{Número de días con precipitación mayor o igual a 20 mm
(R20mm)}\label{nuxfamero-de-duxedas-con-precipitaciuxf3n-mayor-o-igual-a-20-mm-r20mm}

El índice \textbf{R20mm} contabiliza el número de días por año
hidrológico con precipitación diaria igual o superior a 20 mm, asociado
a eventos intensos.

\[
R20mm_j = \sum_{i=1}^{N_j} I(RR_{ij} \ge 20)
\]

\begin{center}\rule{0.5\linewidth}{0.5pt}\end{center}

\subsubsection{Máxima racha de días secos consecutivos
(CDD)}\label{muxe1xima-racha-de-duxedas-secos-consecutivos-cdd}

La métrica \textbf{CDD (Consecutive Dry Days)} corresponde al mayor
número de días secos consecutivos dentro de un año hidrológico,
considerando como día seco aquel con precipitación igual a 0 mm.

\[
CDD_j = \max(L_{dry})
\]

donde \(L_{dry}\) representa la longitud de cada racha consecutiva de
días con \(RR_{ij} = 0\).

\begin{center}\rule{0.5\linewidth}{0.5pt}\end{center}

\subsubsection{Máxima racha de días húmedos consecutivos
(CWD)}\label{muxe1xima-racha-de-duxedas-huxfamedos-consecutivos-cwd}

La métrica \textbf{CWD (Consecutive Wet Days)} corresponde al mayor
número de días húmedos consecutivos dentro de un año hidrológico,
definiendo como día húmedo aquel con precipitación mayor a 0 mm.

\[
CWD_j = \max(L_{wet})
\]

donde \(L_{wet}\) representa la longitud de cada racha consecutiva de
días con \(RR_{ij} > 0\).

\begin{center}\rule{0.5\linewidth}{0.5pt}\end{center}

\subsubsection{Precipitación anual sobre el percentil 95
(R95pTOT)}\label{precipitaciuxf3n-anual-sobre-el-percentil-95-r95ptot}

La métrica \textbf{R95pTOT} corresponde a la suma anual de la
precipitación diaria asociada a eventos extremos moderados, definidos
como aquellos días cuya precipitación supera el percentil 95
(\(P_{95}\)) de la distribución de precipitación diaria positiva del
período de referencia.

\[
R95pTOT_j = \sum RR_{ij}, \quad \text{con } RR_{ij} > P_{95}
\]

\begin{center}\rule{0.5\linewidth}{0.5pt}\end{center}

\subsubsection{Precipitación anual sobre el percentil 99
(R99pTOT)}\label{precipitaciuxf3n-anual-sobre-el-percentil-99-r99ptot}

La métrica \textbf{R99pTOT} representa la suma anual de la precipitación
diaria correspondiente a eventos extremos severos, definidos como
aquellos días con precipitación superior al percentil 99 (\(P_{99}\)) de
la distribución de precipitación diaria positiva del período de
referencia.

\[
R99pTOT_j = \sum RR_{ij}, \quad \text{con } RR_{ij} > P_{99}
\]

\section{3.Resultados}\label{resultados}

\begin{center}\includegraphics{Reporte_files/figure-latex/unnamed-chunk-5-1} \end{center}

\begin{center}\includegraphics{Reporte_files/figure-latex/unnamed-chunk-5-2} \end{center}

\begin{center}\includegraphics{Reporte_files/figure-latex/unnamed-chunk-5-3} \end{center}

\begin{center}\includegraphics{Reporte_files/figure-latex/unnamed-chunk-5-4} \end{center}

\end{document}
